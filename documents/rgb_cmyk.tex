% This file is part of PrinterModel.
% Copyright 2012 David W. Hogg (NYU).

% PrinterModel is free software: you can redistribute it and/or modify
% it under the terms of the GNU General Public License, version 2, as
% published by the Free Software Foundation.

% PrinterModel is distributed in the hope that it will be useful, but
% WITHOUT ANY WARRANTY; without even the implied warranty of
% MERCHANTABILITY or FITNESS FOR A PARTICULAR PURPOSE.  See the GNU
% General Public License for more details.

% You should have received a copy of the GNU General Public License
% version 2 along with this program.  If not, see
% <http://www.gnu.org/licenses/old-licenses/gpl-2.0.html>

% To do
% -----
% - find out what is written about proprietary conversions and calibrations

\documentclass[letterpaper,preprint,dvipdf]{aastex}
\usepackage{natbib}
\usepackage{ifthen}
\newcounter{address}
\newcommand{\latin}[1]{\textit{#1}}
\newcommand{\ie}{\latin{i.e.}}
\newcommand{\eg}{\latin{e.g.}}
\newcommand{\cf}{\latin{cf.}}
\newcommand{\etc}{\latin{etc.}}
\newcommand{\etal}{\latin{et~al.}}

\begin{document}
\title{
  Conversion of color images between RGB and CMYK representations.
}
\author{
  David~W.~Hogg\altaffilmark{\ref{NYU},\ref{email}}
}

\setcounter{address}{1}
\altaffiltext{\theaddress}{\stepcounter{address}\label{NYU}
Center for Cosmology and Particle Physics, Department of Physics, New
York University, 4 Washington Place, New York, NY 10003}
\altaffiltext{\theaddress}{\stepcounter{address}\label{email}
To whom correspondence should be addressed: \texttt{david.hogg@nyu.edu}}

\begin{abstract}
RGB images are described in terms of light emission from a device
(like a screen); CMYK images are described in terms of ink applied to
a white surface (like paper).  The conversion from RGB to CMYK must be
made in taking a screen image and printing it out on a standard
printer.  All the conversion methods in the public domain are
extremely heuristic (and we can hypothesize that most of the
proprietary conversions are too).  Here I propose a conversion method
that is based on a simplified \emph{physical model} for how printer
dyes absorb and reflect incident radiation.  This physical model has
12-ish free hyper-parameters that describe how well each kind of ink
absorbs each color of light.  The conversion is then the optimization
of a pixel-by-pixel model, in which each pixel has four parameters (C,
M, Y, K) and three ``data'' (R, G, B) plus a regularization that
encourages the printer to do as much as possible with the (presumably
cheaper, better) black (K) dye.  The hyper-parameters can be set
heuristically, or by inspection of test prints, or else by
quantitative comparative metrology of a display device and a sensibly
illuminated print-out.
\end{abstract}

\section{Introduction}

In this internet age, the most common standard representation of color
images is red--green--blue (RGB), in which the image is decomposed
into three layers, one corresponding to or indicating the amount of
red light emitted by the image, one corresponding to the amount of
green, and one blue.  The archetypal device to display this image is a
computer screen, which contains red, green, and blue light-emitting
elements.  The three image planes describe how bright the
light-emitting elements ought to be made in displaying the image.  A
zero image is all black; it emits no light at all.  This color
representation is fundamentally ``additive'' because the image
description describes, at each pixel, how much red light to add, how
much green light to add and how much blue light to add.  In the
incoherent limit (in which your screen operates and in which you live,
if you don't illuminate your home small numbers of bright lasers),
electromagnetic radiation is indeed additive in the way imagined by
the implied physical model of the RGB image.  That's good.

Of course the reason that we use three colors to make up images is
because in our color vision, there are three kinds of ``cones'', one
red-sensitive, one green-sensitive, and one blue-sensitive.  There is
much to say here, but one thing that comes to mind is that
\emph{really} perceptual color-space is four-dimensional, because
there are three kinds of cones \emph{and} the rods, which are not
\emph{exactly} in the sensitivity subspace of the cones.  So
three-dimensional color space probably can't quite represent, even in
principle, every perceptual color.  Another comment to make is that
\emph{some} people have \emph{four} kinds of cones; these people can
often feel unsatisfied with colors presented by monitors and printers;
certainly for these individuals, the RGB representation is not
sufficient.  But this note is \emph{not} about physiology or
psychology, it is about printing hardware.

The cyan--magenta--yellow--black (CMYK) representation is designed for
hard-copy printing technology.  The archetypal device is an dye
printer printing on white paper.  The four image planes describe how
much of each of four dyes the printer ought to lay down on the white
paper.  A zero image is all white; it uses no dye at all.  The CMYK
system is usually thought of as ``subtractive'' because dyes absorb
rather than emit light.  However, the system is really
``multiplicative'' because---in the limit of perfect, non-reflective,
purely absorbing dyes applied properly---the absorption of the four
dye layers combine multiplicatively to produce the final absorption in
the image.  That is, each layer of (some particular) ink absorbs some
fraction $f$ of the photons (at some particular wavelength) not
absorbed by the other layers; it changes the reflected light by a
multiplicative factor of $(1-f)$.  The ``subtraction'' of subtractive
dyes is the subtraction of $f$ from 1 in the multiplicative factor (at
least as far as I can tell).

In the printer model below, I am going to assume that the inks are
absorbing in this multiplicative way, but I am going to assume
something much stronger than this as well: I am going to assume that
\emph{all} the ink does is \emph{absorb}.  The toy model to keep in
mind is to a piece of paper with layers of C, M, Y, and K ink on it,
it has a white reflecting surface and then various thicknesses of ink
on top of that surface.  You can think of the ink layers like
superimposed gels or filters or pieces of thin, colored glass.
Incoming light passes through the layers of ink before it gets to the
paper.  At each layer, the incoming light gets partially absorbed (for
example, the cyan C layer absorbs a some fraction of the red light).
The light that makes it through all the layers (in this classical
picture) reflects off the paper and then passes back through all the
layers.  On the way back out, more of the light gets absorbed by the
layers.  The key assumption I am going to make is that the only
possible trajectories for the (classically imagined) photons are:
Either the photon is absorbed on its way in, or else it makes it
through all the layers on the inward journey, bounces off the white
paper layer and is absorbed on the way out, or else it makes it
through all the layers on the inward journey, bounces off the white
paper, and makes it all the way back through all the layers on the
outward journey.  The trajectories I am \emph{not} permitting---the
trajectories I am explicitly assuming are impossible or at least
rare---are those that bounce off of the ink layers directly.  In
reality, if the ink is ``laid down on'' the white surface in
dielectric layers, there will be reflections from the ink surfaces as
well as from the paper surface itself.

The details of the interactions between photons, paper, and ink are
much more complicated (of course) than this model, because the inks
are not laid down in thick dielectric layers on top of the paper but
rather bind with the paper in certain ways; the optical properties of
the dyed surface are extremely non-trivial.  However, the key
assumption is that the inks only augment the \emph{absorption}
properties of the paper only; they don't ever augment the
\emph{reflection} properties of the paper.  This can't be strictly
true for any real printing process, but to parameterize these effects
(which are presumably small), is intellectually and computationally
expensive.

As I note above, in the naive model (paper with layers of ink on
top), light travels through the ink layers twice, the $(1-f)$
absorption factor is the square of the single-pass absorption factor.
In reality, because the ink is not really in a superimposed layer, but
really rather part of the paper, again the story is more complicated,
but we can still represent the absorption process purely
multiplicatively (by assumption!).

One more note: In real printers, when the C, M, Y, or K ink is laid
down at less than full strength, it is usually placed on the paper in
a fine dot pattern.  Each point on the paper has different (in
principle) amounts of C, M, Y, and K ink; it is only when the eye (or
measuring device) averages over a spot larger than the scale of this
pattern that the printing produces the correct colors.  Nothing about
this patterning changes the story below, or the validity of the
assumptions.

I will begin by giving the standard practice and then, in what
follows, re-consider the conversion by thinking about the simplest
possible ``linear'' physical model of dye absorption and a very simple
description of how the inks depart from what you might call
``perfect'' CMYK dyes (to be defined below).  The conversions I obtain
are not quite as simple as the standard conversion, but they produce
much better matches between, \eg, flat-panel computer monitor output
and C-print hardcopy output.  It is worthy of note that the standard
conversions (given below) are wrong \emph{even} for a printer loaded
with perfect CMYK dyes.

One note of caution: the demonstration of color conversions between
screens and hardcopy is somewhat difficult in a paper released on the
internet.  The images and conversions I show are ideally compared (in
2012) on a relatively new computer monitor and a high-quality C-print
(the high-quality digital printing technology at what is now left of
the idea of a photography laboratory).

\section{Conversions}

\paragraph{The standard story:}
The standard non-proprietary conversion of RGB images to CMYK and back
is \latin{ad hoc} and does not consider the physical process of
absorption in a CMYK output.  It is
\begin{eqnarray}\displaystyle
C & = & [1-R] \nonumber \\
M & = & [1-G] \nonumber \\
Y & = & [1-B] \nonumber \\
K & = & \max(C,M,Y) \quad ,
\end{eqnarray}
where we have assumed that the $(R,G,B)$ and $(C,M,Y,K)$ values are on
$[0,1]$ (often they are one-byte integers on [0,255]).  Its reverse is
\begin{eqnarray}\displaystyle
R & = & 1-\max(C,K) \nonumber \\
G & = & 1-\max(M,K) \nonumber \\
B & = & 1-\max(Y,K) \quad .
\label{eq:standard}
\end{eqnarray}
One of the indicators that these conversions are wrong is that when a
good RGB image is converted this way and printed out, it tends to look
bad; there is much advice on the internet about this subject, and it
usually involves tuning the ``gamma'' of the image.  This is a
heuristic fix that adjusts for the fact that these conversions are not
physically justifiable.

DWH: Stuff about detailed printer calibration information; look up
terminology; etc.

\paragraph{A perfect, physical printer:}
In an ideal world, all RGB output devices would have, \eg, red
elements that emit perfectly red light with no ``leakage'' into the
green or blue channels.  All CMYK output devices would have, \eg, cyan
ink that, when laid down with $C=1$, absorbs absolutely all red light
incident on the paper and absorbs absolutely no green or blue light.
While it may be possible to make RGB emission elements that are close
to perfect, in reality it is impossible to make real dyes which, when
applied to real paper, perfectly absorb and transmit as desired.
If such perfect inks were possible, the translation to RGB from CMYK
would be
\begin{eqnarray}\displaystyle
R & = & [1-C]\,[1-K] \nonumber \\
G & = & [1-M]\,[1-K] \nonumber \\
B & = & [1-Y]\,[1-K] \quad .
\label{eq:ideal}
\end{eqnarray}
That is, $C$ and $K$ pigment perfectly blocks---in a multiplicative
way---$R$ light, but $C$ doesn't affect $G$ or $B$ light at all.
Furthermore, $K$ blocks all three colors of light equally.

\paragraph{An imperfect, physical printer:}
In reality, because there is some ``grey'' absorption caused by each
of the $C$, $M$, and $Y$ inks, and because no absorption is perfect,
the translation to RGB from CMYK is something like
\begin{eqnarray}\displaystyle
R & = & [1-C\,(1-\delta_{CR})]\,
        [1-M\,\delta_{MR}]\,
        [1-Y\,\delta_{YR}]\,
        [1-K\,(1-\delta_{KR})] \nonumber \\
G & = & [1-C\,\delta_{CG}]\,
        [1-M\,(1-\delta_{MG})]\,
        [1-Y\,\delta_{YG}]\,
        [1-K\,(1-\delta_{KG})] \nonumber \\
B & = & [1-C\,\delta_{CB}]\,
        [1-M\,\delta_{MB}]\,
        [1-Y\,(1-\delta_{YB})]\,
        [1-K\,(1-\delta_{KB})] \quad ,
\label{eq:real}
\end{eqnarray}
where, for a reasonable output device, all twelve of the
$\delta_{ij}$ coefficients are small.

The twelve small coefficients $\delta_{ij}$ can in general be
determined with radiance experiments on output from the output device;
I would recommend a standard tabulation of all values from all popular
printers!  In practice, in what follows, we will simply make up values
and use them for demonstration purposes.  High-end output users will
want to take some time to estimate these coefficients, either with
metrology or qualitative experiments.

It is worthy of note that even the ``real'' transformation
(\ref{eq:real}) is highly idealized, because it assumes that the dyes
do not reflect light directly (they only absorb or transmit) and that
there are no non-linear effects such as dye mixing or saturation.

(Note: get some real engineering terminology in the above paragraph.)

\section{Implementation}

The implementation of the real transformation (\ref{eq:real}) from
CMYK to RGB is trivial, once the $\delta_{ij}$ have been set.  The
inverse transformation, from RGB to CMYK, requires iteration.  It is
also both over-constrained and under-constrained:  With real
dyes, because the $\delta_{ij}$ are not all zero, some RGB color
combinations cannot be exactly represented in CMYK, so we need to
optimize some non-trivial objective function.  Because there are four
image planes with CMYK and three with RGB, in general the colors that
\emph{can} be represented exactly in CMYK have multiple possible
representations.  I recommend minimizing the objective function
\begin{equation}
\psi = [R-\tilde{R}]^2
     + [G-\tilde{G}]^2
     + [B-\tilde{B}]^2
     + \epsilon\,[1-K]^2 \quad ,
\end{equation}
where $(\tilde{R},\tilde{G},\tilde{B})$ are the ``model'' $(R,G,B)$
values obtained from the $(C,M,Y,K)$ values (the variables, in this
case) and the ``real'' transformation (\ref{eq:real}), and where
$\epsilon$ is set so small that it serves only to break degeneracies.
Minimization of $\psi$ is just least squares, but with an added
$\epsilon$ term that causes the conversion to favor the solution, when
there is a degeneracy, that uses the most black (K) dye and the least
colored dye.

[DWH: In practice, logistic function and L-M optimization.]

[DWH: Note about conversion of bytes to floats: add halfs and divide
  by 256 going byte to float, and multiply by 256 and floor when going
  back.  Nothing else makes sense.]

\section{tests and results}

\acknowledgments It is a pleasure to thank Daniel Foreman-Mackey (NYU)\ldots

\bibliographystyle{apj}
\bibliography{apj-jour,ccpp}

%% \clearpage
%% \begin{figure}
%% \plotone{aok_nearest_lowz.ps}
%% \caption{Same as Figure~\ref{fig:environments}, except using the
%%   transverse, proper, nearest-neighbor distance $\rpmin$ (see text) as
%%   the environment indicator.\label{fig:nearest}}
%% \end{figure}

\end{document}
